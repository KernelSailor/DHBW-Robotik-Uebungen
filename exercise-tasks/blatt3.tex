\documentclass[a4paper,12pt]{article}
\usepackage{fancyhdr}
\usepackage{fancyheadings}
\usepackage[ngerman,german]{babel}
\usepackage[utf8]{inputenc}
%\usepackage[latin1]{inputenc}
\usepackage[active]{srcltx}
\usepackage{algorithm}
\usepackage[noend]{algorithmic}
\usepackage{amsmath}
\usepackage{amssymb}
\usepackage{amsthm}
\usepackage{bbm}
\usepackage{enumerate}
\usepackage{graphicx}
\usepackage{ifthen}
\usepackage{listings}
\usepackage{struktex}
\usepackage{hyperref}
\usepackage{plantuml}
\usepackage{csquotes}

\definecolor{commentgreen}{RGB}{2,112,10}
\definecolor{eminence}{RGB}{108,48,130}
\definecolor{weborange}{RGB}{255,165,0}
\definecolor{frenchplum}{RGB}{129,20,83}
\usepackage{textcomp}

%%%%%%%%%%%%%%%%%%%%%%%%%%%%%%%%%%%%%%%%%%%%%%%%%%%%%%
%%%%%%%%%%%%%% EDIT THIS PART %%%%%%%%%%%%%%%%%%%%%%%%
%%%%%%%%%%%%%%%%%%%%%%%%%%%%%%%%%%%%%%%%%%%%%%%%%%%%%%
\newcommand{\Fach}{Anwendungen der Robotik}
\newcommand{\Name}{Mark Geiger}
\newcommand{\Seminargruppe}{TINF23IT1}
\newcommand{\Matrikelnummer}{}
\newcommand{\Semester}{2025}
\newcommand{\Uebungsblatt}{3} %  <-- UPDATE ME
%%%%%%%%%%%%%%%%%%%%%%%%%%%%%%%%%%%%%%%%%%%%%%%%%%%%%%
%%%%%%%%%%%%%%%%%%%%%%%%%%%%%%%%%%%%%%%%%%%%%%%%%%%%%%

\setlength{\parindent}{0em}
\topmargin -1.0cm
\oddsidemargin 0cm
\evensidemargin 0cm
\setlength{\textheight}{9.2in}
\setlength{\textwidth}{6.0in}

%%%%%%%%%%%%%%%
%% Aufgaben-COMMAND
\newcommand{\Aufgabe}[1]{
  {
  \vspace*{0.5cm}
  \textsf{\textbf{Aufgabe #1}}
  \vspace*{0.2cm}
  
  }
}
%%%%%%%%%%%%%%
\hypersetup{
    pdftitle={\Fach{}: Übungsblatt \Uebungsblatt{}},
    pdfauthor={\Name},
    pdfborder={0 0 0},
    colorlinks=true,
    linkcolor=blue,
    filecolor=magenta,      
    urlcolor=blue,
}

\lstset{
    language=C++,
    frame=tb,
    tabsize=4,
    numbers=left,
    numbersep=3pt, 
    %upquote=true,
    breaklines=true,
    commentstyle=\color{commentgreen},
    keywordstyle=\color{eminence},
    stringstyle=\color{red},
    basicstyle=\color{black}\small\ttfamily, % basic font setting
    emph={int,char,double,float,unsigned,void,bool},
    emphstyle={\color{blue}},
    % keyword highlighting
    otherkeywords={>,<,.,;,-,!,=,~},
    morekeywords={>,<,.,;,-,!,=,~},
    keywordstyle=\color{blue},
}

\title{Übungsblatt \Uebungsblatt{}}
\author{\Name{}}

\begin{document}
\thispagestyle{fancy}
\lhead{\sf \large \Fach{} \\ \small \Name{} \Matrikelnummer{}}
\rhead{\sf \Semester{} \\  Seminargruppe \Seminargruppe{}}
\vspace*{0.2cm}
\begin{center}
\LARGE \sf \textbf{Übungsblatt \Uebungsblatt{}}
\end{center}
\vspace*{0.2cm}

%%%%%%%%%%%%%%%%%%%%%%%%%%%%%%%%%%%%%%%%%%%%%%%%%%%%%%
%% Insert your solutions here %%%%%%%%%%%%%%%%%%%%%%%%
%%%%%%%%%%%%%%%%%%%%%%%%%%%%%%%%%%%%%%%%%%%%%%%%%%%%%%

\Aufgabe{1}

In dieser Übung schreiben wir einen simplen ROS Node der Nachrichten an eine topic sendet. Als basis nutzen wir den Quellcode im Verzeichnis ./code neben dieser Übung. Das Grundgerust hier baut aur der vergangenen Übung 2 auf.

\begin{enumerate}[a)]
\item{
Zunächst erzeugen wir einen neues ROS package, dafür gehen wir ins Verzeichnis ./code/src und führen dann folgenden Befehl aus:
\begin{lstlisting}[language=C++,basicstyle=\fontsize{6.5}{8}\selectfont]
ros2 pkg create --build-type ament_cmake my_robot_controller --dependencies rclcpp geometry_msgs
\end{lstlisting}
}
Dies erzeugt ein ROS Package in der die package.xml mit ein paar Abhängigkeiten vorausgefüllt ist, namentlich rclcpp, geometry\_msgs und ament\_cmake
\item{
Danach erzeugen wir die Datei src/my\_robot\_controller/src/velocity\_publisher.cpp und füllen Sie mit folgendem Inhalt:
\begin{lstlisting}[language=C++,basicstyle=\fontsize{6.5}{8}\selectfont]
#include <rclcpp/rclcpp.hpp>
#include <geometry_msgs/msg/twist.hpp>

class VelocityPublisher : public rclcpp::Node
{
public:
    VelocityPublisher() : Node("velocity_publisher")
    {
        publisher_ = this->create_publisher<geometry_msgs::msg::Twist>("/cmd_vel", 10);
        timer_ = this->create_wall_timer(
            std::chrono::milliseconds(500),
            std::bind(&VelocityPublisher::publish_velocity, this));
    }

private:
    void publish_velocity()
    {
        auto msg = geometry_msgs::msg::Twist();
        msg.linear.x = 0.5;
        msg.angular.z = 0.2;
        RCLCPP_INFO(this->get_logger(), "Publishing: linear.x = %f, angular.z = %f", msg.linear.x, msg.angular.z);
        publisher_->publish(msg);
    }

    rclcpp::Publisher<geometry_msgs::msg::Twist>::SharedPtr publisher_;
    rclcpp::TimerBase::SharedPtr timer_;
};

int main(int argc, char *argv[])
{
    rclcpp::init(argc, argv);
    rclcpp::spin(std::make_shared<VelocityPublisher>());
    rclcpp::shutdown();
    return 0;
}
\end{lstlisting}
Dieser code startet einen Ros node mit dem Namen velocity\_publisher und er sendet alle 500ms eine Nachricht an die /cmd\_vel topic. Dies steuert die simulierten Motoren von unserem Roboter an.
}
\item{
Danach fügen wir die neue Datei in die CMakeLists.txt von unserem Knoten hinzu. Dafür passen wir die Datei an (src/my\_robot\_controller/CMakeLists.txt)
\begin{lstlisting}[language=C++,basicstyle=\fontsize{6.5}{8}\selectfont]
cmake_minimum_required(VERSION 3.5)
project(my_robot_controller)

find_package(ament_cmake REQUIRED)
find_package(rclcpp REQUIRED)
find_package(geometry_msgs REQUIRED)

add_executable(velocity_publisher src/velocity_publisher.cpp)
ament_target_dependencies(velocity_publisher rclcpp geometry_msgs)

install(TARGETS
  velocity_publisher
  DESTINATION lib/${PROJECT_NAME}
)

ament_package()
\end{lstlisting}
}
\item{
Jetzt erweitern wir noch unsere launch Datei, damit auch unser neu geschriebener Knoten gestartet wird! Ergänzen in der Datei src/my\_robot\_bringup/launch/my\_robot.launch.xml folgendes:
\begin{lstlisting}[language=C++,basicstyle=\fontsize{6.5}{8}\selectfont]
<launch>
  <node
    name="velocity_publisher"
    pkg="my_robot_controller"
    exec="velocity_publisher"
    output="screen" />
</launch>
\end{lstlisting}
}
\item{Dann musst du einmal neu kompilieren und die erzeugte setup.bash sourcen. Du solltest inzwischen wissen wie das geht!}
\item{Jetzt können wir unsere launch datei starten:
\begin{lstlisting}[language=C++,basicstyle=\fontsize{6.5}{8}\selectfont]
ros2 launch my_robot_bringup my_robot.launch.xml
\end{lstlisting}
}
\item{Wir sehen nun unseren altbekannten Roboter der automatisch eine Kreisbahn abfährt. Experimentiere ein bisschen mit dem code des Knoten um andere Bewegungen zu realisieren!}
\end{enumerate}
%%%%%%%%%%%%%%%%%%%%%%%%%%%%%%%%%%%%%%%%%%%%%%%%%%%%%%
%%%%%%%%%%%%%%%%%%%%%%%%%%%%%%%%%%%%%%%%%%%%%%%%%%%%%%
\end{document}

